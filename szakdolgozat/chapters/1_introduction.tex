\Chapter{Bevezetés}

% A fejezet célja, hogy a feladatkiírásnál kicsit részletesebben bemutassa, hogy miről fog szólni a dolgozat.
% Érdemes azt részletezni benne, hogy milyen aktuális, érdekes és nehéz probléma megoldására vállalkozik a dolgozat.

% Ez egy egy-két oldalas leírás.
% Nem kellenek bele külön szakaszok (section-ök).
% Az irodalmi háttérbe, a probléma részleteibe csak a következő fejezetben kell belemenni.
% Itt az olvasó kedvét kell meghozni a dolgozat többi részéhez.

A tesztvezérelt fejlesztés (Test Driven Development) napjaink egyik legismertebb szoftverfejlesztési folyamata a professzionális programozásban. Ez a folyamat nagyon rövid fejlesztési ciklusokon alapul: a követelmények nagyon specifikus tesztesetekként vannak megfogalmazva, a kódot pedig ahhoz mérten írjuk, hogy az át fog így menni a teszten. Ez teljes ellentettje a hagyományos szoftverfejlesztésnek, mivel az megengedi azon kódrészleteket is, amelyek nem felelnek meg a követelményeknek teljesen.

Ez a módszer nagyon előnyösnek bizonyult a szoftverfejlesztés területén. Erre a módszerre épül a tesztvezérelt programozás oktatás elve is. Ezzel teljesen vagy részben automatizáltan lehet a programozás iránt érdeklődő embereket tanítani és visszajelzést adni nekik. Napjainkban ez egy rendkívül felkapott témakör lett, több cég építette már ki erre a teljes portfólióját, és rengeteg weboldal, alkalmazás készült már ami a manuális, emberi erőforrásokon alapuló programozás oktatást szeretné leváltani, megkönnyíteni azáltal, hogy egy teljesen automatizáltan működő tanulóprogramot ad a programozást tanulni kívánók kezébe.

A dolgozat a programozás oktatás esetében vizsgálja azt, hogy a tesztek készítésének, az azokkal történő számonkérésnek milyen előnyei, esetlegesen hátrányai lehetnek. A dolgozat keretében készül egy program, amely bemutatásra kerül. A programban programozási feladatokat és a hozzájuk tartozó teszteseteket, az elemzéshez és egyéb megjelenítéshez szükséges metaadatokat lehet elhelyezni, a tanulók pedig szerkeszthetik majd ellenőrizhetik a feladatokhoz tartozó programkódokat. A program lefuttatja a programot és a feladathoz tartozó teszteseteket. Elemzi a feltöltött forráskódot, és ez alapján értékeli a programot, figyelmezteti a felhasználót a gyakran előforduló hibákra, illetve javaslatot ad a programkód javítására. A nem ismert eseteket továbbítja a mentorok számára manuális átnézésre.
