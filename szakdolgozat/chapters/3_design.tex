\Chapter{Tervezés}

% Itt kezdődik a dolgozat lényegi része, úgy értve, hogy a saját munka bemutatása.
% Jellemzően ebben szerepelni szoktak blokkdiagramok, a program struktúrájával foglalkozó leírások.
% Ehhez célszerű UML ábrákat (például osztály- és szekvenciadiagramokat) használni.

% Amennyiben a dolgozat inkább kutatás jellegű, úgy itt lehet konkretizálni a kutatási módszertant, a kutatás tervezett lépéseit, az indoklást, hogy mit, miért és miért pont úgy érdemes csinálni, ahogyan az a későbbiekben majd részletezésre kerül.

% Ebben a fejezetben az implementáció nem kell, hogy túl nagy szerepet kapjon.
% Ez még csak a tervezési fázis.
% (Nyilván ha olyan a téma, hogy magának az implementációnak a módjával foglalkozik, adott formális nyelvet mutat be, úgy a kódpéldákat már innen sem lehet kihagyni.)

\Section{Követelmény specifikáció}

\SubSection{Szerepkörök}

\begin{itemize}
    \item[--] látogató
    \item[--] tanuló
    \item[--] mentor
    \item[--] adminisztrátor
\end{itemize}

\SubSection{Magas szintű követelmények}

\subsubsection{Felhasználói fiók}

\begin{itemize}
    \item[--] A felhasználónak tudnia kell regisztrálni felhasználói fiókot.
    \item[--] A felhasználónak tudnia kell bejelentkeznie a már létrehozott felhasználói fiókjába.
    \item[--] A felhasználónak tudnia kell szerkesztenie a saját profilját.
    \item[--] Az adminisztrátornak tudnia kell módosítani mások profilját és jogkörét módosítani.
\end{itemize}

\subsubsection{Feladat}

\begin{itemize}
    \item[--] A felhasználónak tudnia kell listáznia az elérhető feladatokat.
    \item[--] A felhasználónak tudnia kell megtekinteni a kilistázott feladatokat.
    \item[--] A tanulónak tudnia kell a kilistázott feladatokat megoldania a webalkalmazásban.
    \item[--] Az alkalmazásnak tudnia kell a feladathoz tartozó automatizált teszteseteket lefuttatni.
    \item[--] Az alkalmazásnak tudnia kell a feladathoz tartozó forráskódelemző modult lefuttatni.
    \item[--] Az alkalmazásnak tudnia kell a feladathoz tartozó forráskód-reprezentáló modult lefuttatni.
    \item[--] A tanulónak tudnia kell megtekinteni az előzőleg leadott megoldásait, és az ahhoz tartozó hozzászólásokat.
    \item[--] A mentornak tudnia kell a tanulók megoldásait megtekinteni.
    \item[--] A mentornak tudnia kell a tanulók megoldásaihoz hozzászólásokat írnia.
    \item[--] A mentornak tudnia kell a korábbi hozzászólásokat megtekinteni.
    \item[--] A tanulónak tudnia kell hozzászólást írni a saját megoldott feladataihoz.
    \item[--] Az adminisztrátornak tudnia kell törölni hozzászólásokat.
\end{itemize}

\SubSection{Alacsony szintű követelmények}

\subsubsection{Felhasználói fiók}

\textbf{\textit{Regisztráció}}

A webalkalmazás megnyitása után a látogatók regisztrálhatnak felhasználói fiókot. A fiók regisztrálásához  szükség van egy felhasználónév és jelszó párosra.

A felhasználónév alfanumerikus karaktereket, és egyedülálló kötőjeleket tartalmazhat. A felhasználónév nem kezdődhet vagy végződhet kötőjellel. A felhasználónév legfeljebb 32 karakter hosszú lehet. A felhasználónév minden regisztrált felhasználónak egyedi.

A jelszónak legalább 8 karakter hosszúnak kell lennie illetve legalább egy számot, egy kis- és nagybetűt kell tartalmaznia.

A regisztráció után a felhasználó automatikusan kap egy egyedi azonosítót, illetve megkapja a Tanuló szerepkört, majd bejelentkeztetésre kerül, és átirányítódik az alkalmazás főoldalára.

\textbf{\textit{Bejelentkezés}}

A webalkalmazás megnyitása után a látogatók bejelentkezhetnek a már beregisztrált felhasználói fiókukba.

A bejelentkezéshez szükség van a felhasználó és jelszó párosra. A hibás adatokra a felület figyelmeztet, de biztonsági szempont miatt a szerver csak egy általános hibaüzenetet ad vissza és nem közli a hiba pontos okát azaz, hogy melyik adat hibás vagy esetleg nem is létezik az a felhasználó.

A bejelentkezés után a felhasználó átirányítódik az alkalmazás főoldalára.

\textbf{\textit{Fiók szerkesztése}}

A bejelentkezett felhasználók szerkeszthetik a saját fiókjukat. A fiók szerkesztése opció a felhasználóknak a navigációs sávról elérhető.

Az adminisztátor szerepkörrel rendelkező felhasználók minden felhasználó fiókját szerkesztheti. 

A fiók szerkesztése aloldalon a felhasználók megváltoztathatják a felhasználónevüket és a jelszavukat. Ezenkívül az adminisztrátorok az adott felhasználó szerepkörét is megváltoztathatják.

A regisztrációnál leírt validációs folyamatoknak itt is le kell futnia.

\subsubsection{Feladat}

\textbf{\textit{Feladatok listázása}}

A felhasználók listázhatják a webalkalmazásba feltöltött feladatokat. A listázott feladatok szűrhetők több szempont alapján is. Többek között nehézség (könnyű, közepes, nehéz), nyelv (C\#, JavaScript), és egyéni előrehaladás (nem próbált, próbált, teljesített feladat) szerint lehet szűrni.

Listázás, és opcionálisan szűrés után a felhasználók a feladatok bővebb adatait megjeleníthetik egy előnézeti felületen, de rögtön a feladatmegoldó felületre is ugorhatnak a preferált programozási nyelvet kiválasztva.

\textbf{\textit{Feladat megtekintése}}

Az előnézeti felületen láthatják a felhasználók az adott feladat minden adatát, és felhasználó-specifikus adatokat is. Például a korábban leadott megoldásokat, vagy a feladat státuszát.

Itt a felhasználók a preferált programozási nyelvet kiválasztva átléphetnek a feladatmegoldó felületre.

\textbf{\textit{Feladat megoldása}}

A feladatmegoldó felület fő eleme a kódszerkesztő ablak, amelyben betöltéskor a feladathoz tartozó kezdő forráskód szerepel vagy amennyiben már jártak ezen a felületen és módosítottak a kezdő forráskódon azon forráskódjuk szerepel itt. 

Ezenkívül a felületen még szerepel a feladatleírás, a programfuttatás és tesztesetek eredményei, a forráskódelemző javaslatai illetve amennyiben már oldották meg ezt a feladatot abban az esetben a korábbi megoldásaik is szerepelnek itt.

A felület 2 darab függőlegesen elválasztott panelra van felosztva, ahol jobb oldalon szerepel a kódszerkesztő, a bal oldalon pedig az előzőekben említett adatok.

A felhasználók le tudják futtatni az általuk megírt forráskódot, mintha egy fejlesztői környezetben lennének. A "Futtatás" gombra kattintva elküldi a szervernek amely feldolgozza azt.

A szerver a felhasználó forráskódját összepárosítja a megfelelő programozási nyelvvel és az ahhoz tartozó tesztesetekkel. Ezután lefordítja a kódot, lefuttatja a teszteseteket, ha ezek sikeresek voltak akkor pedig a forráskód elemzés és reprezentálás is lefuttatásra kerül.

A szerver ezeket az adatokat visszaadja válaszban a felhasználónak és ez kliens oldalon megjelenítődik számára. Fordítási hiba esetén megkapja a felhasználó a pontos hibaüzenetet, sikeres fordítás esetén pedig a tesztesetek eredményeit is láthatja. A felhasználó látja a teszteseteket neveit, leírását, az eredményét, a hibaüzenetet és a konzolra kiíratott üzeneteket (amiket hibakereséshez tud felhasználni).

Ha minden teszteset sikeresen lefutott akkor a felhasználó látja a forráskódelemző modul eredményét azaz az automatizált javaslatokat. Ezeket később a hozzászólások felületen is megtekintheti.

A forráskód-reprezentáló modul eredményét nem kapja meg a felhasználó, ez letárolódik az adatbázisban, viszont ez csak a mentorok számára látható.

A "Beadás" gombra kattintva a feladatot leadhatja a felhasználó, viszont ez csak akkor lehetséges ha minden teszteset lefutott hiba nélkül. Ekkor a mentorok számára is láthatóvá válik a megoldás, akik tudnak hozzászólásokat fűzni a megoldásokhoz. Illetve a felhasználó számára is átvált megoldott státuszba, így könnyen nyomonkövethető, hogy melyik feladattal végzett már. Beadás után akár megoldhatja újra a feladatot, új megoldást is adhat be.

\textbf{\textit{Teszteset futtató modul}}

\begin{itemize}
    \item[--] Lefordítja a felhasználó által beadott forráskódot és a feladathoz tartozó teszteseteket.
    \item[--] Lefuttatja a teszteseteket.
    \item[--] Feldolgozva visszaadja az eredményeket.
\end{itemize}

\textbf{\textit{Forráskód elemző modul}}

\begin{itemize}
    \item[--] Elemzi a felhasználó által beadott forráskódot. Automatikusan felismeri a hibákat, javítható részeket és ezekre javaslatot ad.
    \item[--] A javaslatok szövegeit egyedi azonosítókulcsuk alapján az adatbázisból nyeri ki.
    \item[--] Minden feladathoz és azonbelül programozási nyelvhez külön elemző szubmodul tartozik.
\end{itemize}

\textbf{\textit{Forráskód-reprezentáló modul}}

\begin{itemize}
    \item[--] Normalizálja a felhasználó által beadott forráskódot, egy standardizált, egységesített alakra hozza.
    \item[--] A normalizált alak segítségével megvizsgálhatóak a hasonló programkódok, így lehet csoportosítani őket. A csoportosított kódok segítségével nem kell a mentoroknak a nagyon hasonló kódokat többször értékelni hanem egyszerre letudhatják, amivel időt spórolhatnak meg.
\end{itemize}

\textbf{\textit{Korábbi megoldások megtekintése}}

A felhasználók megtekinthetik a korábban leadott megoldásaikat, és az ahhoz tartozó hozzászólásokat.

A megoldások programozási nyelv szerint elkülönülnek, és alapértelmezetten csak az legutóbb leadott megoldásuk jelenik meg, de böngészhetnek a korábbi megoldásaik között is.

A mentorok és a felhasználók egyaránt hozzászólhatnak a megoldásokhoz. A tanulók csak a saját megoldásait látják, viszont a mentorok egy külön felületen feladatonként és nyelvenként az összes megoldás között böngészhetnek, illetve megtekinthetik a forráskód reprezentációkat is, és csoportosíthatják eszerint.

Az adminisztrátorok törölhetik a felhasználók hozzászólásait.



% Külön szakaszban érdemes részletesen kitérni az elkészítendő alkalmazással kapcsolatos követelményekre.
% Ehhez tartozhatnak forgatókönyvek (\textit{scenario}-k).
% A szemléletesség kedvéért lehet hozzájuk képernyőkép vázlatokat is készíteni, vagy a használati eseteket más módon szemléltetni.

% ...

% \Section{Táblázatok}

% Táblázatokhoz a \texttt{table} környezetet ajánlott használni.
% Erre egy minta \aref{tab:minta}. táblázat.
% A hivatkozáshoz az egyedi \texttt{label} értéke konvenció szerint \texttt{tab:} prefixszel kezdődik.

% \begin{table}[h]
% \centering
% \caption{Minta táblázat. A táblázat felirata a táblázat felett kell legyen!}
% \label{tab:minta}
% \begin{tabular}{l|c|c|}
% a & b & c \\
% \hline
% 1 & 2 & 3 \\
% 4 & 5 & 6 \\
% \hline
% \end{tabular}
% \end{table}

% \Section{Ábrák}

% Ábrákat a \texttt{figure} környezettel lehet használni.
% A használatára egy példa \aref{fig:cimer}. ábrán látható.
% Az \texttt{includegraphics} parancsba 
% Az ábrák felirata az ábra alatt kell legyen.
% Az ábrák hivatkozásához használt nevet konvenció szerint \texttt{fig:}-el célszerű kezdeni.

% \begin{figure}[h]
% \centering
% \includegraphics[scale=0.3]{images/me_logo.png}
% \caption{A Miskolci Egyetem címere.}
% \label{fig:cimer}
% \end{figure}

% \Section{További környezetek}

% A matematikai témájú dolgozatokban szükség lehet tételek és bizonyításaik megadására.
% Ehhez szintén vannak készen elérhető környezetek.

% \begin{definition}
% Ez egy definíció
% \end{definition}

% \begin{lemma}
% Ez egy lemma
% \end{lemma}

% \begin{theorem}
% Ez egy tétel
% \end{theorem}

% \begin{proof}
% Ez egy bizonyítás
% \end{proof}

% \begin{corollary}
% Ez egy tétel
% \end{corollary}

% \begin{remark}
% Ez egy megjegyzés
% \end{remark}

% \begin{example}
% Ez egy példa
% \end{example}
